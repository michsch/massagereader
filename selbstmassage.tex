\chapter{Selbstmassage}

\section{Strecken}
\begin{enumerate}
  \item aufrecht hinstellen
  \item Strecken: "`Nach den Kirschen greifen"'
  \item nach unten beugen und Arme baumeln lassen
\end{enumerate}

\section{Nacken und Schultern}
Diese Übungen kann man im Stehen oder im Sitzen ohne Rückenlehne und ohne Armlehnen machen.

Jeden Schritt erst auf einer Seite machen, dann auf der anderen:
\begin{enumerate}
  \item mit der Hand über die gegenüberliegende Schulter streichen: vom Haaransatz bis zum Ellenbogen, leicht wieder zurück (3x)
  \item mit den Fingerspitzen kleine, feste Kreisbewegungen neben der Halswirbelsäule: nach oben arbeiten bis zum Haaransatz
  \item mit lockerer Faust rhythmisch auf die Schultermuskeln klopfen
\end{enumerate}

Abschluss für beide Seite zusammen: mit beiden Händen leichte Streichungen: von den Seiten des Gesichts am Kinn entlang (da kreuzen sich die Wege) über die Schultern bis zu den Fingerspitzen (nach Belieben wiederholen)

\section{Arme}
Für die Armmassage am besten hinsetzen und den zu massierenden Arm aufs Bein stützen.

Erst einen Arm komplett massieren, dann den anderen:
\begin{enumerate}
  \item mit der Handfläche kräftig vom Handgelenk zur Schulter streichen, sanft zurück (wiederholen)
  \item den Arm von unten nach oben durchkneten
  \item mit dem Daumen kräftige Kreisbewegungen auf der Vorderseite des Unterarms machen
  \item Mulden um den Ellenbogen kreisförmig mit Daumen und Fingern bearbeiten
  \item mit der Handfläche gegen den Oberarm klopfen
  \item Abschluss: leicht über den ganzen Arm streichen
\end{enumerate}

\section{Hände}
Für die Handmassage am besten hinsetzen und die zu massierende Hand auf dem entsprechenden Bein ablegen.

Erst eine Hand komplett massieren, dann die andere:
\begin{enumerate}
  \item mit festen Druck über den Handrücken zum Handgelenk streichen
  \item Hand zwischen Handballen und Fingern (der anderen Hand) zu\-sam\-men\-drü\-cken
  \item jeden einzelnen Finger massieren:
    \begin{enumerate}
      \item die einzelnen Fingerglieder zwischen Daumen und Zeigefinger zu\-sam\-men\-drü\-cken
      \item die Fingergelenke mit kleinen kreisenden Bewegungen eines Fingers massieren
      \item am Finger ziehen, dabei leicht drehen
    \end{enumerate}
  \item mit dem Daumen zwischen den Sehnen am Handrücken entlangstreichen: von den Knöcheln zum Handgelenk (4x)
  \item Hand umdrehen (die Handfläche liegt jetzt oben)
  \item mit dem Daumen kräftige Kreisbewegungen auf der Handfläche ausführen
  \item mit dem Daumen die Handfläche punktweise drücken
  \item mit dem Daumen auf dem Handgelenk Druckmassage ausführen
  \item Abschluss: mit der Handfläche über die Handfläche der massierten Hand streichen: von den Wurzeln der Finger bis zum Handgelenk, dann mit dem Ballen der Hand fest gegen die Handfläche drücken und wieder zurückgleiten (wiederholen)
\end{enumerate}

\section{Gesicht gegen Spannungskopfschmerzen}

Dass ihr diese Massage richtig macht, merkt ihr daran, dass die Kopfschmerzen direkt nachlassen.

\begin{enumerate}
  \item Drückt die Haut an der \fett{Nasenwurzel mit Daumen und Zeigefinger} von links und rechts fest zusammen.
  \item Wandert nacheinander beide Augenbrauen von der Nasenwurzel in Richtung der Ohren entlang und \fett{drückt die Augenbrauen mit Daumen und Zeigefinger} von links und rechts zusammen.
  \item Wandert mit den Fingerspitzen in einem \fett{großen Kreis in beiden Augenhöhlen} entlang und massiert dabei auf der Stelle \fett{in kleinen Kreisen mit viel Druck}. Achtet dabei darauf, dass ihr nur auf den Knochen drückt, aber nicht auf dem Augapfel.
\end{enumerate}
